%
% page layout
%
% Äußerer Seitenrand = one inch + \hoffset 
\hoffset = 0pt
% Oberer Seitenrand = one inch + \voffset
\voffset = -1cm
% Abstand zwischen äußerem Seitenrand und Text
% auf ungeraden Seiten
\oddsidemargin = 0pt
% Abstand zwischen äußerem Seitenrand und Text
% auf ungeraden Seiten
\oddsidemargin = 0pt
% Abstand zwischen oberem Seitenrand und Header
\topmargin = 0pt
% Höhe des Headers der ersten Seite
\headheight = 151pt
% Abstand zwischen Header und Text
\headsep = 0pt
% Texthöhe
\textheight = 205mm
% Textbreite
\textwidth = 170mm
% Abstand vom Text zu den Marginalien
% \marginparsep = 11pt 10 
% Breite der Marginalien
% \marginparwidth = 54pt
% Abstand Text zu Unterkante Footer
% \footskip = 30pt 
% Papierbreite
%\paperwidth = 597pt 
% papierhöhe
%\paperheight = 845pt

% Einzug von Absätzen
\parindent 0mm
% Abstand von Absätzen
\parskip .6\baselineskip plus 1pt
%
% commands
%
\renewcommand{\bottomfraction}{1}
\renewcommand{\topfraction}{1}
\renewcommand{\textfraction}{0}
\textfloatsep1ex plus 1ex minus.5ex
%
% avoid date
%
\date{}
%
% to get more tolerance (mir)
%
\tolerance800
\emergencystretch2em
\doublehyphendemerits5000
\hfuzz0pt
\leftskip0pt minus 1pt
\rightskip0pt minus 1pt

%\setlength\parskip{.4\baselineskip plus5pt minus2pt}

%
%  neuer pagestyle
%
% % % % % % % % % % % % % % % % % % % % % % % % % % %
%
% Header

% gemeinsame Infos:
\newcommand{\halbjahr}{{\bf Summer Semester 2011}}
\newcommand{\studiengangi}{Automotive Systems}
\newcommand{\studiengangii}{}
\newcommand{\studiengangiii}{}
\newcommand{\semesteri}{ASM-SB}
\newcommand{\semesterii}{}
\newcommand{\semesteriii}{}
\newcommand{\fach}{ Reliable Embedded Systems }
\newcommand{\lecture}{{\bf Real Time System Design }}
\newcommand{\fachnummer}{}
\newcommand{\hilfsmittel}{closed book apart from \par 2 manually written sheets of paper DIN-A4}
\newcommand{\dozent}{Friedrich}
\newcommand{\dauer}{60 minutes}

\newlength{\headerspaltenbreite}
\setlength{\headerspaltenbreite}{8.5cm}

% Header für die erste Seite
\newcommand{\firstpageheader}{
{\bfseries Hochschule Esslingen \hfill\hfill Fakulty Graduate School}\\
\begin{small}
{ % Änderung lokal halten
% Platz zwischen \hline und Text einfügen
\setlength{\extrarowheight}{1.5pt}
\begin{tabular}{|lm{\headerspaltenbreite}|ll|}\hline
\multicolumn{2}{|l|}{\halbjahr} & Page No.:  & \thepage\ of \anzblaetter \\\hline
Programme:   & \studiengangi  & Semester:   & \semesteri   \\\hline
%               & \studiengangii &             & \semesterii  \\
%               & \studiengangiii&             & \semesteriii \\\hline
Module:  & \fach          &             & \\
Lecture: & \lecture       & Lecturer:     & \dozent      \\\hline
Mode:   & \hilfsmittel   & Duration:      & \dauer       \\\hline
Name:          &                & Student number: &          \\[3ex]\hline               
\end{tabular}
}
\end{small}
}
% Header für erste Seite setzen
\fancypagestyle{firstpagestyle}{
   \fancyhf{}
   \chead{\firstpageheader}
   \cfoot{} %\cfoot{\thepage}
}
%\thispagestyle{firstpagestyle}

% Header für die folgenden Seiten
\newcommand{\nextpageheader}{
\begin{small}
{ % Änderung lokal halten
% Platz zwischen \hline und Text einfügen
\setlength{\extrarowheight}{1.5pt}
\begin{tabular}{|lm{9.6cm}|ll|}\hline
\multicolumn{2}{|l|}{\halbjahr} & Page No.:  & \thepage\ of \anzblaetter \\\hline
Lecture:  & \lecture          & Semester: & \semesteri  \\\hline
%Name:          &                & Student number: & \\[3ex]\hline               
\end{tabular}
}
\linebreak
\linebreak
\end{small}
}

\pagestyle{fancy}
% Header leeren
\fancyhf{}
\chead{\nextpageheader}
\cfoot{} %\cfoot{\thepage}
%\cfoot{\small\it Bitte geben Sie alle Aufgabenblätter wieder ab!}
% kein weiterer horizontaler Strich
\renewcommand{\headrulewidth}{0pt}
\renewcommand{\plainheadrulewidth}{0pt}
