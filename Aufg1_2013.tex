\aufgabe{}{1}

What are typical functions a real-time computer system must perform?
\begin{unteraufgaben}
\item Calculate the temporal accuracy of the system if the crankshaft revolves with 6000 rpm.
\end{unteraufgaben}

\aufgabe{}{1}

 What does signal conditioning mean? Give an example

\aufgabe{}{2}

 Explain the term “error containment region”.
\aufgabe{}{2}

 Define the notions of offset, drift, drift rate, precision, and accuracy.

\aufgabe{}{2}

What are the basic techniques for error detection?

\aufgabe{}{4}

How can clock synchronization assist in finding the primary event of an alarm shower?

\pagebreak
\headheight = 78pt

\aufgabe{}{10}

Explain the three different types of orders with regard to alarms in a distributed real-time
system. Which of the orders implies another?


\aufgabe{}{3}

A quadrocopter has a maximum acceleration (downwards) when all engines are stopped.
With what sampling rate do we have to scan the acceleration sensor when we want to make
sure that the deviation between the real-time entity and real-time image of the relative
position in space is less than 2 mm, i. e. the quadrocopter moves by at most 2 mm during a
sampling interval. Assume that initially the quadcopter is hovering (v = 0).


\aufgabe{}{5}

Explain the difference between state correction and rate correction for a clock. What are the
advantages and disadvantages for each method?

\aufgabe{}{5}

What is a hidden channel? Define the notion of permanence.

\pagebreak


\aufgabe{}{5}

What is the difference between a state observation and an event observation? Discuss their
advantages and disadvantages.




\aufgabe{}{5}

 What is a ground state in a real-time computer system?
\pagebreak

\aufgabe{}{5}

What are the basic techniques for error detection? Compare ET systems and TT system
from the point of view of error detection.

\aufgabe{}{5}

Assume a computer system that can control three concurrently operating trains running on a
model railway track, containing 5 switches and 12 signals.
Identify the h-state at the reintegration point. Which part of the h-state can be enforced on
the environment at the reintegration point? What is the minimal remaining h-state at the
reintegration point?

\aufgabe{}{5}

Explain the terms fault, error, an failure.

\aufgabe{}{5}

Compare the efficiency of event-triggered and time-triggered communication protocols at
low load and peak load.

\aufgabe{}{5}

Given a bandwidth of 10 MBits/sec, a channel length of 500 m and a message length of 48
bits, what is the maximum protocol efficiency that can be implemented by the media access
level of a bus system?


\aufgabe{}{5}

Consider a PAR protocol with a bus transport delay of 1 msec, an acknowledgement
detection timeout of 3 msec, 2 retries and a negligible computation time on each node. What
is the maximum protocol jitter this system can exhibit?

\aufgabe{}{5}

 Estimate the average and worst-case response time of a TTP/C system with 8 FTUs, each
one consisting of two nodes that exchange messages with 5 data bytes on a channel with a
bandwidth of 1 Mbit/sec. Assume that the interframe gap is 8 bits.

\aufgabe{}{5}

How is the consistency of the data transfer across the CNI enforced by the TTP protocol?.

\aufgabe{}{5}
Calculate the data efficiency of a TTP/A system that consists of 10 nodes where each node
sends periodically a three byte message (user data). Assume that the intermessage gap
between the Fireworks byte and the first data byte is 4 bitcells, and the intermessage gap
between two successive data bytes is two bitcells. The gap between the end of one round
and the start of the next round is 12 bitcells.

