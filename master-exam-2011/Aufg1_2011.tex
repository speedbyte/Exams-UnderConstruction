\aufgabe{}{1}

Sketch a simple model of a real-time system and partition it into three important parts. Name
each cluster and the interfaces between them.

\aufgabe{}{1}

What is the difference between a real-time image and a real-time entity?


\aufgabe{}{2}

Hard disk drives are often advertised with very high MTBFs. Some manufacturers have
switched to a different specification, the annual failure rate (AFR). The AFR is defined as
that percentage of a large number of disc drives that exhibit a defect when continuously
being run for a year.
What is the MTBF for a hard disc drive when the observations of the last year gave an AFR
of 0.73%, that is 0.73% of all disc drives that had been running continuously for a year
exhibited a defect?


\aufgabe{}{2}

Is real-time computing equivalent to fast computing? What is the main goal in the design of
real-time computing systems? Please discuss briefly.


\aufgabe{}{2}

 The start of an injection in a modern combustion engine must be controlled to better than $1\deg$,
to adhere to environmental standards. The idle motor speed of an engine is 800 rpm, the
maximum rate is 4500 rpm. The maximum change in motor speed in case of a load change
is 1500 rpm/sec.
a) Compute the maximum temporal accuracy for the start of the injection.
b) Compute for a one-cylinder four cycle engine at constant rpm the minimum time that is
available for the computation of the injection time. A four cycle engine needs to inject
only every other full crankshaft turn. Assume that there are no other processes running
on the computer calculating the injection time.
c) When the injection time is calculated early, it can become imprecise due to a change in
rpm during this time. Compute how far in advance the injection time can be calculated
without violating the $1\deg$ limit, assuming a maximum motor speed change. Assume for
your calculation that the computation doesn’t take any time at all.


\aufgabe{}{4}

Explain the three different types of orders with regard to alarms in a distributed real-time
system. Which of the orders implies another?


\pagebreak
\headheight = 78pt

\aufgabe{}{10}

How can a sparse time base help to avoid agreement protocols?


\aufgabe{}{3}

Assume you have node clocks running at a frequency of 100 MHz. What precision can you
achieve when measuring a time interval, where the start event and the end event can origin
from different nodes, and the node clocks are synchronized to a global clock?

\aufgabe{}{5}

Given a resynchronization period of 500 msec, and a clock drift rate of $10^{-6}$ sec/sec, what
latency jitter can be tolerated to achieve a precision of $20\mu$sec using the FTA algorithm in a
system with 5 clocks where 1 clock could be malicious ($\mu$(5,1) = 1,5)?


\aufgabe{}{5}

Please explain the difference between a simple task (S-task) and a complex task (C-task).
\pagebreak


\aufgabe{}{5}

 What is a history state? Please explain with an example.

\aufgabe{}{5}

Calculate the action delay in a distributed system with the following parameters:
dmax=20msec, dmin=5msec,
and a) no global time available, granularity of local time is 100 $\mu$sec
and b) global time with granularity of 500 $\mu$sec.

\pagebreak

\aufgabe{}{5}

When is a set of nodes replica determinate?


\aufgabe{}{5}

What is state estimation?

\aufgabe{}{5}

Explain the terms fault, error, an failure.

\aufgabe{}{5}

Given a bandwidth of 10 MBits/sec, a channel length of 200 m and a required protocol
efficiency of 90%, what is the maximum message length in bit that can be implemented by
the media access level of a bus system


\aufgabe{}{5}

What mechanism can lead to thrashing? How should you react in an event-triggered system
if thrashing is observed?


\aufgabe{}{5}

How can one distinguish between a Fireworks byte and a data byte in the TTP/A protocol?

\aufgabe{}{5}

 Calculate the data efficiency of a TTP/A system that consists of 5 nodes where each node
sends periodically a two byte message (user data). Assume that the intermessage gap
between the Fireworks byte and the first data byte is 4 bitcells, and the intermessage gap
between two successive data bytes is two bitcells. The gap between the end of one round
and the start of the next round is 6 bitcells.

\aufgabe{}{5}

Explain three major problems that we encounter in interrupt-driven software.

