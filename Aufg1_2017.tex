


\aufgabe{}{10}

According to the NASA Managers, the Space shuttle Challenger had a failure rate of 1 in 100000 hours. However, Richard Feynman who was investigating the explosion of the space shuttle that led to the killing of all crew members, reported a failure rate of 1 in 200 hours.

\begin{enumerate}
\item Calculate the MTTF as per the NASA managers?
\item Calculate the MTTF as per Richard Feynman?
\item Calculate the reliability of the space shuttle 1 hour after the shuttle is switched on as per the NASA managers?
\item Calculate the reliability of the space shuttle 1 hour after the shuttle is switched on as per Richard Feynman?
\item Why is it important to carefully calculate the MTTF for highly critical systems?
\end{enumerate}

\begin{tikzpicture}
\calcremainingheight
\node (rect) at (0,0) [draw, text width=16.6 cm, minimum height=\remainingheight]{};
\node[below right, text width=16.6 cm] at (rect.north west) {
    \lsg{
%    MTTF = $1/\lambda$ \\
%    \lambda_{Nasa Managers} = $1/100000$ hours. Hence MTTF = 100000 hours\\
%    \lambda_{Feynman} = $1/200$ hours. Hence MTTF = 200 hours. \\\\
%    The MTTF of the system calculate according to Feynman is quite small and hence such a system should not have been launched. For a safety critical system, the MTTF should be ideally in 100 thousands of hours. \\
%    Reliablility R(t) = $\exp^{-\lambda(t-t\_{0}}$ \\
%    Reliablity_{Nasa Managers}(1) = $\exp^{-(1/100000)*(1-0}$ \\
%    Reliablity_{Feynman}(1) = $\exp^{-(1/200)*(1-0}$ \\
    }
   };
\end{tikzpicture}

\pagebreak

\aufgabe{}{20}

The US division of Mercedes \& Bosch teamed up to create Robo-Taxis ( driverless taxis ) in April this year. There are innumerous challenges in driverless cars.
One such challenge is driving in the city with lots of pedesterians around. Your job is to focus on the system design, where you can detect the movement of the pedesterians,
and correspondingly steer / brake or accelerate the car. 

\begin{enumerate}

\item Depict the design of your system with the help of 3 components of the real-time systems.
\item Define hard and soft deadlines through examples. Discuss if your designed system should have hard deadlines? If so, why?
\item List some of your system's functional and non-functional requirement. Please elaborate the temporal requirement falling under the category of non-functional requirements.
\item What kind of sensors would you employ to detect the pedesterians? 
\item How can you monitor the failure of components such as sensors after the car has rolled out of production?

\end{enumerate}

\begin{tikzpicture}
\calcremainingheight
\node (rect) at (0,0) [draw, text width=16.6 cm, minimum height=\remainingheight]{};
\node[below right, text width=16.6 cm] at (rect.north west) {
    \lsg{
    Please sketch a model here
    }
   };
\end{tikzpicture}

\newpage

Contd...

\begin{tikzpicture}
\calcremainingheight
\node (rect) at (0,0) [draw, text width=16.6 cm, minimum height=\remainingheight]{};
\node[below right, text width=16.6 cm] at (rect.north west) {
    \lsg{
    Please sketch a model here
    }
   };
\end{tikzpicture}

\pagebreak

Nodes with identical sensors are replicated to improve the reliability of the whole system, so that even if one of the sensor fails, the system will still continue working based on the other sensors. 

\begin{enumerate}
\item Draw a block diagram to show how these different nodes are communicating via a communication bus.
\item Explain the difference between state messages and event messages. Mention some state messages with respect to your system.
\end{enumerate}


\begin{tikzpicture}
\calcremainingheight
\node (rect) at (0,0) [draw, text width=16.6 cm, minimum height=\remainingheight]{};
\node[below right, text width=16.6 cm] at (rect.north west) {
    \lsg{
    Please sketch a model here
    }
   };
\end{tikzpicture}

\pagebreak

Following are 4 dependant tasks and 1 independent task to realise your software.
The semantic for the task set is ( Release time, Computation Time, Time Period )
\begin{description}
\item[$\bullet$ Task A] : Sensor acquisition task ( 0, 50, 100 ) 
\item[$\bullet$ Task B] : Sensor processing task ( 0, 20, 100 ) 
\item[$\bullet$ Task C] : Communication Task ( 0, 20, 100 )
\item[$\bullet$ Task D]: Decision Task ( 0, 10, 100 )
\end{description}
A -> B -> C -> D ( 
Dependancy graph )
\begin{description}
\item[$\bullet$ Task E] : Independent Task: Diagnostic Task ( 0, 50, 3000 )
\end{description}

\begin{enumerate}
\item Compute the Hyperperiod of all the task set.
\item Calculate the processor utilisation factor ( Ratio of computation and time period of all tasks ) for the combined task sets.
\item Discuss if its possible to run the above task set in a single core system? If not, then how many cores do you need? 
\end{enumerate}


\begin{tikzpicture}
\calcremainingheight
\node (rect) at (0,0) [draw, text width=16.6 cm, minimum height=\remainingheight]{};
\node[below right, text width=16.6 cm] at (rect.north west) {
    \lsg{
    Please sketch a model here
    }
   };
\end{tikzpicture}

\pagebreak

\aufgabe{}{10}

In 2017, ten atomic clocks ( all manufactured from SpetraTime ) failed to oscillate in the European Union’s Galileo navigation satellites.
The possible cause is thought to be waking up the satellites after deep sleep which leads to a short circuit in the clock circuitry.
 
\begin{enumerate}
\item What is meant by oscillations in a clock?
\item What role does a clock play in the powermanagement of a CPU?
\item Assuming a clock drift at the rate of $10^{-5}$ sec/sec, what would be the offset of the clock after 1 hour?
\item In an ensemble of 4 clocks with a latency jitter of $5*10^{-5}$ seconds, the system designer wants to maintain a precision of 36 ms. In what intervals should the clock be synchronised?
\item Assuming 4 identical clocks are employed from the same manufacturer in your design, and one of the clock is Byzantine. Calculate the synchronisation interval to maintain the required precision.
%$R_{int}$ = 
\end{enumerate}

\begin{tcolorbox}[height fill, title=Your solution]
    \lsg{
    Please sketch a model here
    }
\end{tcolorbox}

%\begin{tikzpicture}
%\calcremainingheight
%\node (rect) at (0,0) [draw, text width=16.6 cm, minimum height=\remainingheight]{};
%\node[below right, text width=16.6 cm] at (rect.north west) {
%    \lsg{
%    Please sketch a model here
%    }
%   };
%\end{tikzpicture}

\pagebreak

\aufgabe{}{10}
\begin{enumerate}

\item Draw the different layers of OSI Reference Model.
\item In a bus communication, discuss 2 methods, how the bus access is resolved at the time when two nodes want to send their data at the same time?
\item For a message length of 100 bytes, bandwidth of 1 Mbit/second and the distance between two nodes as 10 meters, calculate the bus efficiency. Assume speed of light in the medium as 2/3rd of speed of light.
\item How does the bus efficiency change, if the bandwidth is 1 Gbit/second?
\item How can you improve the bandwidth of the transmission from 1 Mbit/second to for example 10 Mbit/second through software methods? 
\end{enumerate}

\begin{tikzpicture}
\calcremainingheight
\node (rect) at (0,0) [draw, text width=16.6 cm, minimum height=\remainingheight]{};
\node[below right, text width=16.6 cm] at (rect.north west) {
    \lsg{
    Please sketch a model here
    }
   };
\end{tikzpicture}

\pagebreak

\begin{tikzpicture}
\calcremainingheight
\node (rect) at (0,0) [draw, text width=16.6 cm, minimum height=\remainingheight]{};
\node[below right, text width=16.6 cm] at (rect.north west) {
    \lsg{
    Please sketch a model here
    }
   };
\end{tikzpicture}


\aufgabe{}{10}
\begin{enumerate}

\item What is a context switch?
\item What is the process of context switch between a running task and an interrupt?
\item Is the process of context switch between a running task and an interrupt the same as a context switch between two tasks?
\item What triggers a task switch in a round robin scheduling algorithm?
\item What is one major advantage of a multi core system over a single core system with respect to context switches?
\end{enumerate}

\begin{tikzpicture}
\calcremainingheight
\node (rect) at (0,0) [draw, text width=16.6 cm, minimum height=\remainingheight]{};
\node[below right, text width=16.6 cm] at (rect.north west) {
    \lsg{
    Please sketch a model here
    }
   };
\end{tikzpicture}

\newpage


\begin{tcolorbox}[height fill, title=Your solution]
    \lsg{
    Please sketch a model here
    }
\end{tcolorbox}

\newpage


\begin{tikzpicture}
\calcremainingheight
\node (rect) at (0,0) [draw, text width=16.6 cm, minimum height=\remainingheight]{};
\node[below right, text width=16.6 cm] at (rect.north west) {
    \lsg{
    Please sketch a model here
    }
   };
\end{tikzpicture}
