\aufgabe{}{1}

One important parameter for many real-time computer programs is the WCET. Why is it
difficult to obtain the WCET? Give as many reasons as you can think of.

\aufgabe{}{1}

Explain the term “composability”. Give examples.

\aufgabe{}{2}

Explain the term “error containment region”.

\aufgabe{}{2}

What types of failures can a physical clock exhibit? Please explain briefly.

\aufgabe{}{2}

What are the basic techniques for error detection?

\aufgabe{}{4}

 What are the advantages of having a global time available on nodes in a distributed real-
time system with regard to interval measurements, action delay, and alarm root cause
detection?

\pagebreak
\headheight = 78pt

\aufgabe{}{10}

Explain the three different types of orders with regard to alarms in a distributed real-time
system. Which of the orders implies another?


\aufgabe{}{3}

What is an agreement protocol? Why would one try to avoid it?

\aufgabe{}{5}

Explain the difference between state correction and rate correction for a clock. What are the
advantages and disadvantages for each method?

\aufgabe{}{5}

Given a resynchronization period of 1000 msec, and a clock drift rate of $10^{-6}$ sec/sec,
what precision can be achieved in case of a latency jitter of 15$\mu$sec using the FTA
algorithm in a system with 5 clocks where 1 clock could be malicious ($\mu$(5,1) = 1,5)?

\pagebreak


\aufgabe{}{5}

 Please explain the difference between an instant and an event.


\aufgabe{}{5}

What is a ground state in a real-time computer system?

\pagebreak

\aufgabe{}{5}

Assume we have a distributed system with the following parameters: dmax=20msec,
dmin=5msec. What granularity does your time base (local or global) need if you want to
keep the action delay below 50 msec? Consider the following two cases:
a) global time available
b) no global time


\aufgabe{}{5}

What are the temporal obligations of clients and servers at a client-server interface in a real-
time system?


\aufgabe{}{5}

Explain the difference between a parametric and a phase-sensitive RT image. How can you
create parametric RT images?


\aufgabe{}{5}

What is the relationship between action delay and temporal accuracy?


\aufgabe{}{5}

Given a bandwidth of 5 MBits/sec, a channel length of 300 m and a message length of 40
bits, what is the maximum protocol efficiency that can be implemented by the media access
level of a bus system?

\aufgabe{}{5}

 Consider a PAR protocol with a bus transport delay of 1 msec, an acknowledgement
detection timeout of 3 msec, 2 retries and a negligible computation time on each node. What
is the maximum protocol jitter this system can exhibit?


\aufgabe{}{5}

What is the purpose of the bus guardian in a TTP/C controller? How is it controlled?

\aufgabe{}{5}

What is the membership service in the TTP/C protocol? Explain its purpose and briefly
explain how it works.


\aufgabe{}{5}

 Calculate the data efficiency of a TTP/A system that consists of 8 nodes where each node
sends periodically a three byte message (user data). Assume that the intermessage gap
between the Fireworks byte and the first data byte is 3 bitcells, and the intermessage gap
between two successive data bytes is two bitcells. The gap between the end of one round
and the start of the next round is 8 bitcells


