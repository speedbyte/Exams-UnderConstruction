\documentclass[a4paper,12pt,ngerman]{article}
\usepackage[utf8]{inputenc}
\usepackage[T1]{fontenc}     % Um die Zeichen korrekt zu kodieren
\usepackage[ngerman]{babel}
\usepackage{amssymb}  % für \Box
\usepackage{fancyhdr}
\usepackage{setspace}
\usepackage{framed}
%\usepackage{xcolor}
\usepackage{hhline}
\usepackage{longtable}
\usepackage{array}     % für m{} etc. in Tabellen
%\usepackage{booktabs}  % für \addlinespace[2ex] in Tabellen
\usepackage{graphicx}
\usepackage[export]{adjustbox}
\usepackage{caption}
\usepackage{subcaption}
%\usepackage{hyperref}  % für \autoref{...}
\usepackage{lastpage}  % für thelastepage im Header
\usepackage{paralist}  % compactenum in Unteraufgaben
\usepackage{enumitem}  % Anpassbare Enumerates/Itemizes
%\usepackage{pstricks}  % Tex-Graphiken exportiert von Dia
\usepackage{tikz}    
\usepackage{xparse}    % für neue Befehle mit variabler Anzahl von Parametern (hier \lsg)
\usepackage{soul}      % zum "verstecken" von Lösungstext
%\usepackage[plainpages=false, pdfpagelabels,colorlinks=true, pdfstartview=FitV, linkcolor=blue, citecolor=blue, urlcolor=blue]{hyperref}
\usepackage{listings}
\usepackage{eurosym}   % für das Euro-Symbol
\usepackage{verbatim}
%\usetikzlibrary{positioning,shapes,shadows,arrows.meta,calc}
\usepackage{lipsum}
\usepackage[most]{tcolorbox}

% % % % % % % % % % % % % % % % % % % % % % % % % % % % %
% Befehle zum Ein- und Ausblenden von Lösungen
%

\ifloesung
	\NewDocumentCommand\lsg{+m +g}{%
		\textcolor{red}{#1}
	}
	\newcommand{\nlsg}[1]{}
\else
	\NewDocumentCommand\lsg{+m +g}{%
		\IfNoValueF{#2}
			{#2}
			{}
	}
	\newcommand{\nlsg}[1]{#1}
\fi

% Jedes Zeichen innerhalb von \geheim{...} entfernen
% wenn die Zeichen durch etwas anderes (z.B. ?) ersetzt
% werden sollen dann \phantom{\the\SOUL@token} ersetzen 
% (z.B. durch ?)
% benötigt \usepackage{soul}
\makeatletter
\DeclareRobustCommand{\geheim}{%
  \SOUL@setup
  \def\SOUL@everytoken{%
    \phantom{\the\SOUL@token}}\SOUL@}
\makeatother

% % % % % % % % % % % % % % % % % % % % % % % % % % % % %

% % % % % % % % % % % % % % % % % % % % % % % % % % % % %
% Konfiguration der Codeausgabe mit listings
%
%\usepackage{pxfonts}  % erlaube \ttfamily in bold, geht derzeit nicht in meinem MikTex unter Windows  
\renewcommand{\ttdefault}{pcr} % Courier font auswählen, der bold erlaubt, (Alternative zu pxfonts)
\lstset{basicstyle=\ttfamily}
%\lstset{keywordstyle=\bfseries} % ist Default
% \bf hinzufügen, wenn es bold sein soll
% gilt auch für Code im Text
\lstset{keywordstyle=\bfseries}
%\lstset{keywordstyle=\bfseries\ttfamily\underbar}
%\lstset{keywordstyle=\color{blue}\ttfamily}
\lstset{stringstyle=\it}
%\lstset{stringstyle=\color{red}\ttfamily}
%\lstset{commentstyle=\ttfamily\itstyle}
%\lstset{commentstyle=\color{green}\ttfamily}
\lstset{tabsize=4}
\lstset{showtabs=false}
\lstset{language=C++}
\lstset{morekeywords={ostringstream, istringstream, stringstream, ostream}}
%\lstset{showspaces=false, 
%        showtabs=false, tab= , 
%		 keywordstyle=\blue\bfseries, 
%		 commentstyle=\it\color{greenf},%
%        showstringspaces=false, framexleftmargin=5mm, 
%		 frame=none, numbers=none, numberstyle=\tiny, 
%		 stepnumber=1, numbersep=5pt,%
%        texcl=true,escapechar=!
%}
% automatischen Zeilenumbruch erlauben  \lstset{breaklines=true}  
% automatischer Zeilenumbruch nur bei Whitespace              
\lstset{breakatwhitespace=false}
\lstset{showstringspaces=false,
        commentstyle=\color{black} 
%        morecomment=[l]{//}
}
% Coder der Lösung in rot oder ausgeblendet
\ifloesung
\lstset{morecomment=[l][\color{red}]{//=},
		morecomment=[s][\color{red}]{//+}{//-},
		morecomment=[s][\color{red}]{//l\{}{//l\}},		
		morecomment=[s][\color{red}]{/*l\{*/}{/*l\}*/}
}
\else
% mit [is] wird der Kommentar ignoriert und nicht ausgegeben ==> Platz für Lösung einplanen
% [il] funktioniert nicht, löscht alles Folgende
\lstset{morecomment=[is]{//=}{.},
	 	morecomment=[is]{//+}{//-},
		morecomment=[is]{//l\{}{//l\}},	 	
	 	morecomment=[is]{/*l\{*/}{/*l\}*/}
}
\fi
% Umlaute in Listings zu erlauben
\lstset{literate=      
{Ö}{{\"O}}1       
{Ä}{{\"A}}1
{Ü}{{\"U}}1
{ß}{{\ss}}2
{ü}{{\"u}}1
{ä}{{\"a}}1
{ö}{{\"o}}1
{~}{$\sim$}{1}     % hochgestellte Tilde in eine in Mitte
                   % gestellte Tilde umwandeln
}
% Programmcode im Text
\newcommand\pct[1]{\lstinline!#1!}

% % % % % % % % % % % % % % % % % % % % % % % % % % % %

% % % % % % % % % % % % % % % % % % % % % % % % % % % %
% Konfiguration für tikz
%

%\definecolorset{rgb}{}{}%
%{lightyellow,1,1,0.6;%
% lightblue,0.529,0.81,1;%
% lightred,1,0.6,0.6;%
% greenf,0,0.5647,0%
%}

%\pgfrealjobname{pruefung}
%\usetikzlibrary{shapes,arrows,intersections,backgrounds}
%\usetikzlibrary{decorations.markings, decorations.pathmorphing,patterns,snakes}
%\usetikzlibrary{circuits.ee.IEC, circuits.logic.IEC}
%tikzstyle{ST2style} = [auto, line width=1pt, >=stealth]

%\tikzstyle{input}    = [coordinate]
%\tikzstyle{output}   = [coordinate]

%\tikzstyle{blockS}   = [fill=lightyellow, draw=black, rectangle, minimum height=4ex, minimum width=3em] % Block Strecke ; normal hoch
%\tikzstyle{blockBS}  = [style = blockS,  minimum height=6ex] % Block Strecke ; höher für Bruch
%\tikzstyle{blockR}   = [style = blockS,  fill=cyan!20]       % Block Regler ; normal hoch
%\tikzstyle{blockBR}  = [style = blockR, minimum height=6ex] % Block Regler ; höher für Bruch

%\tikzstyle{sumS}     = [draw=black, fill=lightyellow, circle, inner sep=0pt,minimum size=3mm]  % Summationsstelle Strecke
%\tikzstyle{sumR}     = [style = sumS, fill=cyan!20]  % Summationsstelle Regler

%\tikzstyle{connectiondot} = [draw=black, fill=black, circle, inner sep=0pt,minimum size=1mm]  % Punkt zum Verbinden von Signallinien
%\tikzstyle{connectiondotvector} = [style = connectiondot, minimum size=2mm]  % Punkt zum Verbinden von Signallinien

%\tikzstyle{vectorline} = [line width=1pt, double]
%\tikzstyle{pinstyle} = [pin edge={to-,ST2style, black}]

% % % % % % % % % % % % % % % % % % % % % % % % % % % % %

\renewcommand{\bf}{\bfseries}

\newcommand{\anzblaetter}{\pageref{LastPage}}

% % % % % % % % % % % % % % % % % % % % % % % % % % % % %
% Definition des Aufgabenlayouts und -nummerierung
%

% Der Zähler aufgabe als Unterzähler von chapter
% zählt die Aufgaben in einem Kapitel
% Mit \aufgabe{Titel} wird eine neue Aufgabe begonnen.
\newcounter{aufgabe}
\setcounter{aufgabe}{0}
\renewcommand{\theaufgabe}{\arabic{aufgabe}}
\newenvironment{aufgabe}[2]%
	{\refstepcounter{aufgabe}
    \vskip 6pt plus 3pt minus 3pt
    \ifenglisch
      {\bf\large Question \arabic{aufgabe}: #1 \hfill (#2 Min.)}
    \else
      {\bf\large Aufgabe \arabic{aufgabe}: #1 \hfill (#2 Min.)}
    \fi
   	%\\
   	%\vskip 3pt plus 3pt minus 3pt	
	}%
	{}
% Umgebung für Liste der Unteraufgaben
\newlist{aufgabenliste}{enumerate}{3}
\setlist[aufgabenliste]{topsep=0pt,partopsep=0pt,itemsep=0pt,parsep=4pt}
\setlist[aufgabenliste,1]{label=\theaufgabe.\arabic*}
\setlist[aufgabenliste,2]{label=\alph*}
\setlist[aufgabenliste,3]{label=\roman*}
\newenvironment{unteraufgaben}%
    { \begin{aufgabenliste} }%
    { \end{aufgabenliste} }

% % % % % % % % % % % % % % % % % % % % % % % % % % % % %

% % % % % % % % % % % % % % % % % % % % % % % % % % % % %

% Text für englische Sprache für bilinguale Klausuren hervorheben
\newcommand{\engl}[1]{\emph{#1}}

% Ein paar Variablen im Logfile ausgeben
\newcommand{\vars}{%
\message{****hoffset = \the\hoffset}
\message{****voffset = \the\voffset}
\message{****headheight = \the\headheight}
}

%\newdimen\remainingheight
%\newcommand*{\calcremainingheight}{%
%    \remainingheight\dimexpr\pagegoal-\pagetotal-\baselineskip-\parskip
%}

\newdimen\remainingheight
\newcommand*{\calcremainingheight}{%
    \ifdim\pagegoal=\maxdimen
        \remainingheight\dimexpr\textheight-0.4pt\relax
    \else
        % edit 2: replaced -\baselineskip by -\lineskip-0.4pt
        % edit 3: removed -\parskip
        \remainingheight\dimexpr\pagegoal-\pagetotal-\lineskip-0.4pt-\parskip\relax
    \fi
}

