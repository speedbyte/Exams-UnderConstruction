\aufgabe{}{1}

 Give an example for an end-to-end protocol in the context of a distributed real-time system.
Why would you have to use an end-to-end protocol at the interface between a computer
system and the controlled object?

\aufgabe{}{1}

State three functional requirements for real-time systems.

\aufgabe{}{2}

An average car is operated about 400 hours per year. Compute the permissible MTTF if one
out of one thousand cars may fail to provide the requested service throughout that year.
Would such a car be considered a system with an ultrahigh reliability requirement? Please
explain.

\aufgabe{}{2}

Why can there be conflicts between reliability and maintainability? Please explain.

\aufgabe{}{2}

Discuss the advantages and disadvantages of an event-triggered communication system vs.
a time triggered communication system. What would you prefer for ultra-dependable
systems, and why?

\aufgabe{}{4}

Explain the two failure modes of a clock.

\pagebreak
\headheight = 78pt

\aufgabe{}{10}

 How many binary digits (bits) would a digital counter need, if it was to directly measure a
time interval of 1 hour with a digitization error of less than 10 nsec?

\aufgabe{}{3}

Assume you need to measure a time interval with a precision of 10 nsec, where the start
event and the end event can origin from different nodes, and the node clocks are
synchronized to a global clock. What frequency must your node clocks at least have?


\aufgabe{}{5}

 Given a latency jitter of 10 $\mu$sec, a resynchronization period of 100 msec, and a clock drift
rate of $10^{-6}$ sec/sec, what precision can be achieved by the FTA algorithm in a system with
5 clocks where 1 clock could be malicious ($\mu$(5,1) = 1,5)?

\aufgabe{}{5}

Please explain why it is difficult to determine the WCET of processes in a hard-real time
system. State three methods that have been advised to determine WCET, and discuss their
limitations.

\pagebreak


\aufgabe{}{5}

 What is an observation in the context of real-time systems? Explain the two major types of
observations and discuss their advantages and disadvantages.

\aufgabe{}{5}

Calculate the action delay in a distributed system with the following parameters:
d$_{max}$=10msec, d$_{min}$=2msec,
and a) no global time available, granularity of local time is 50 $\mu$sec
and b) global time with granularity of 100 $\mu$sec.


\pagebreak

\aufgabe{}{5}

What kind of redundancy would you employ if you needed to detect errors caused by
independent software faults and by transient and permanent physical hardware faults?


\aufgabe{}{5}

What is triple modular redundancy (TMR)?

\aufgabe{}{5}

Fault tolerance can be implemented by two fail-silent nodes or by Triple Modular
Redundancy (TMR). Discuss the advantages and disadvantages of each approach.

\aufgabe{}{5}

Given a bandwidth of 1 MBits/sec, a channel length of 100 m and a message length of 64
bits, what is the limit of the protocol efficiency that can be achieved at the media access
level of a bus system?

\aufgabe{}{5}

Explain the role of the three time-outs in the ARINC 629 protocol. Is it possible for a collision
to occur on an ARINC 629 bus?

\aufgabe{}{5}

What mechanism helps to ensure the fail-silence of a TTP controller in the temporal
domain?

\aufgabe{}{5}

 Estimate the average and worst-case response time of a TTP/C system with 8 FTUs, each
one consisting of two nodes that exchange messages with 10 data bytes on a channel with a
bandwidth of 10 MBit/s. Assume that the interframe gap is 8 bits.

\aufgabe{}{5}

Explain the difference between polling and sampling.
